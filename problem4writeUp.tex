\documentclass[11pt]{amsart}
%\pagenumbering{gobble}
\usepackage{amsmath}
\usepackage{amsfonts}
\usepackage{amssymb}
\usepackage{tikz}
\renewcommand*{\proofname}{Answer}

%% Change space between paragraphs and indentation.
\setlength\parskip{\medskipamount} 
\setlength\parindent{2em}  
%% New environments for problems.
\newenvironment{problem}[1]
{\par \noindent \ignorespaces \textbf{Problem #1:}}
{}

\renewenvironment{part}[1]
{\par \noindent \ignorespaces \textbf{(#1) \quad }}
{}

%%%%%%%%%%%%%%  MACROS BELOW %%%%%%%%%%%%%%%%%
%% All your favorite rings and fields.
\newcommand{\NN}{\mathbb N}
\newcommand{\ZZ}{\mathbb Z}
\newcommand{\QQ}{\mathbb Q}
\newcommand{\RR}{\mathbb R}
\newcommand{\CC}{\mathbb C}
%% Mathy things you want upright.
\DeclareMathOperator{\Ker}{Ker}
\DeclareMathOperator{\Id}{Id}
\DeclareMathOperator{\Inv}{Inv}
\DeclareMathOperator{\GCD}{GCD}
%% For use with matrices.
\newcommand{\Matrix}[4]{ \left( \begin{array}{cc}  #1 & #2 \\  #3 & #4 \\ \end{array} \right) }
\newcommand{\Zero}{\mathbf{0}}
\newcommand{\One}{\mathbf{1}}
%% Variations on equality
\newcommand{\isom}{\cong}
%% Shortcuts for arrows.
\newcommand{\From}{\colon}
\newcommand{\To}{\rightarrow}
%%%%%%%%%%%%%%%%%%%%%%%%%%%%%%%%%%%%%%%%%%

%%  The start of the document.
\title{CSE 142:  Homework 2}
\author{Jerico Factor  \and Roberto Oregon \and Marco Rodriguez}
\date{October 23, 2019}  % Change to the due date.

\begin{document}
\maketitle

\section*{Problems}

\begin{problem}{2.2} 
What is the $L_1$ distance between the first instance of the test set and the first instance of train set? 
\end{problem}
\begin{proof} 
2811
\end{proof}

\begin{problem}{2.3} 
What is the $L_2$ distance between the first instance of the test set and the first instance of train set? 
\end{problem}
\begin{proof} 
2750.45
\end{proof}

\begin{problem}{2.4} 
What is the $L_1$ distance between the first instance of the test set and the first instance of train set? 
\end{problem}
\begin{proof} 
2750
\end{proof}

\begin{problem}{2.5} What are the labels of the 5 nearest neighbors (in order) of the first instance in the test set when using $L_2$ distance (left to right being the closest neighbor to the farthest neighbor).
\end{problem}
\begin{proof} $ -1, -1, 1, -1, 1$
\end{proof} 

\begin{problem}{2.6} In this question you will be experimenting with different values of K. List the predictions for every instance of test set for $K = 1, 3,4$ and 720 with $L_2$ distance measure. As an answer to this question complete the table below. 
\end{problem}
\begin{proof}See Table 1 below.
\end{proof} 
\begin{table}[h]
\caption{for 2.6}
\begin{tabular}{ccccc}
\hline
\multicolumn{1}{|c|}{Test Instance} & \multicolumn{1}{c|}{K = 1, $L_2$} & \multicolumn{1}{c|}{K = 3, $L_2$} & \multicolumn{1}{c|}{K = 5, $L_2$} & \multicolumn{1}{c|}{K = 720, $L_2$} \\ \hline
1                                   & -1                             & -1                             & -1                             & -1                               \\
2                                   & 1                              & 1                              & 1                              & -1                               \\
3                                   & 1                              & 1                              & -1                             & -1                               \\
4                                   & 1                              & 1                              & 1                              & -1                               \\
5                                   & 1                              & 1                              & -1                             & -1                               \\
6                                   & -1                             & 1                              & 1                              & -1                               \\
7                                   & -1                             & -1                             & -1                             & -1                               \\
8                                   & 1                              & 1                              & 1                              & -1                               \\
9                                   & 1                              & 1                              & 1                              & -1                               \\
10                                  & -1                             & -1                             & -1                             & -1                              
\end{tabular}
\end{table}
\begin{problem} {2.7} Now we will study of effect of using different types of distance measures for a fixed K. List the predictions for every instance of test set for $L_1$,  $L_2$, and $L_ \infty$ distance measures and $K = 9$. As an answer to this question, complete the table below.
\end{problem}
\begin{proof}See Table 2 below.
\end{proof} 
\begin{table}[h]
\caption{for 2.7}
\begin{tabular}{cccc}
\hline
\multicolumn{1}{|c|}{Test Instance} & \multicolumn{1}{c|}{$L_1$} & \multicolumn{1}{c|}{$L_2$} & \multicolumn{1}{c|}{$L_\infty$} \\ \hline
1                                   & 1                       & 1                       & 1                         \\
2                                   & 1                       & -1                      & -1                        \\
3                                   & -1                      & -1                      & -1                        \\
4                                   & 1                       & 1                       & 1                         \\
5                                   & -1                      & -1                      & -1                        \\
6                                   & 1                       & 1                       & 1                         \\
7                                   & 1                       & 1                       & 1                         \\
8                                   & -1                      & -1                      & -1                        \\
9                                   & -1                      & -1                      & 1                         \\
10                                  & -1                      & -1                      & -1                       
\end{tabular}
\end{table}

\end{document}



































